\documentclass[12pt]{article}
\usepackage{epsfig}
\usepackage{times}
\renewcommand{\topfraction}{1.0}
\renewcommand{\bottomfraction}{1.0}
\renewcommand{\textfraction}{0.0}
\setlength {\textwidth}{6.6in}
\hoffset=-1.0in
\oddsidemargin=1.00in
\marginparsep=0.0in
\marginparwidth=0.0in                                                                               
\setlength {\textheight}{9.0in}
\voffset=-1.00in
\topmargin=1.0in
\headheight=0.0in
\headsep=0.00in
\footskip=0.50in                                         
\setcounter{page}{1}
\begin{document}
\def\pos{\medskip\quad}
\def\subpos{\smallskip \qquad}
\newfont{\nice}{cmr12 scaled 1250}
\newfont{\name}{cmr12 scaled 1080}
\newfont{\swell}{cmbx12 scaled 800}
%%%%%%%%%%%%%%%%%%%%%%%%%%%%
\begin{center}
{\large\bf
PHYS 20323/60323: Fall 2020 - LaTeX Example}\\
\end{center}
%%%%%%%%%%%%%%%%%%%%%%%%%%%%%%

\noindent1.  Consider a particle confined in a two-dimensional infinite square well        %       
\[                     
{V}(x,y) = \left\{
			\begin{array}{ll}
                  		\textnormal{0,} & \textnormal{if} \ 0 \le {x}\le {a} \ , \  0 < y < a\\
                 		\infty, & \textnormal{otherwise}
                \end{array}             
    \right.
\]
\\
\indent The eigenfunctions have the form:\\
\[
\Psi \textnormal{(x,y)} = \frac{2}{a}sin(\frac{n\pi x}{a})sin(\frac{m\pi y}{a})\\
\]
\indent with the corresponding energies being given by:\\
\[
E_{mn} = (n^{2} + m^{2})\frac{\pi ^{2} \hbar ^{2}}{2ma^{2}}
\]
\\
\indent \ \ (a) (5 points) What are the levels of degeneracy of the five lowest energy values?     \   %                              
\\    
\indent \ \ (b) (5 points) Consider a  perturbation given by:     \\
\[
\hat{H}^\prime = a^{2}V_{0}\delta(x-\frac{a}{2})\delta(y-\frac{a}{2})
\]                   
\indent \ \ \ Calculate the first order correction to the ground state energy.
\\
\\
{\noindent\bf 2. The following questions refer to stars in the Table below:\\}  
 \indent Note: There may be multiple answers\\
\indent\begin{tabular}{|c|c|c|c|c|c|}\hline
Name & Mass            & Luminosity         & Lifetime                  & Temperature & Radius \\\hline
Zeta   & $60. \ M_{sun}$ & $10^{6} \ L_{sun}$ & $8.0 \times 10^{5} \ years$ &  \            & \  \\\hline
Epsilon & $6.0 \ M_{sun}$ & $10^{3}\ L_{sun}$ & \ & 20,000 K & \ \\\hline
Delta & $2.0 \ M_{sun}$& \ & $5.0 \times 10^{8} \ years$ & \ & $2 \ R_{sun}$ \\\hline
Beta &$1.3 \ M_{sun}$&$3.5 \ L_{sun}$& \ & \ & \ \\\hline
Alpha&$1.0\  M_{sun}$& \ & \ & \ & $1\ R_{sun}$ \\\hline
Gamma&$0.7 \ M_{sun}$& \ & $4.5 \times 10^{10} \ years$ & 5000 K & \ \\\hline
\end{tabular}
\\
\\
\\
\indent \ \ (a) (4 points) Which of these stars will produce a planetary nebula at the end of their life        
                       \\
\\
\\
\\
\indent \ \ (b) (4 points)  Elements heavier than \emph{Carbon} will be produced in which stars.        %                       



%%%%%%%%%%%%%%%%%%%%%%%%%%%%%%%%%%%%%%%%%%%%%%%%%%%%%%%%%%%%

\end{document}